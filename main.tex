\documentclass{article}
\usepackage{graphicx}

\begin{document}

\begin{center}
    {\Huge be nam khoda} \\[1cm]
    {\LARGE projeye kargah} \\[0.5cm]
    {\large Mohammad Matin Roozbehani} \\[0.5cm]
    {\large 402531387} \\[1cm]
    \hrule
\end{center}

\vspace{1cm}

\section*{fehrest mataleb}

\section{Git and GitHub}
\subsection{Repository Initialization and Commits}
\subsection{GitHub Actions for LaTeX Compilation}

\section{Exploration Tasks}
\subsection{Vim Advanced Features}
\subsection{Memory Profiling}
\subsubsection{Memory Leak}
\subsubsection{Memory Profilers}
\subsection{GNU/Linux Bash Scripting}
\subsubsection{fzf}
\subsubsection{Using fzf to find your favorite PDF}
\subsubsection{Opening the file using Zathura}

\section{Git and FOSS}
\subsection{README.md}
\subsection{Issues}
\subsection{FOSS contribution}





\newpage








\section*{taklif}

\textbf{1.1 Repository Initialization and Commits}

\textbf{Setting Up the Repository} \\
To set up the repository, I first went to GitHub and created a new repository named Final-Assignment. During the repository creation, I disabled the Initialize this repository with a README option. After creating the repository, I used the git clone command to clone it to my local system.

\begin{verbatim}
git clone https://github.com/username/Final-Assignment.git
cd Final-Assignment
\end{verbatim}

\textbf{Creating the LaTeX File and Making the First Commit} \\
Inside the local repository, I created a new file called assignment.tex and added initial content. I then added the changes to the staging area using git add and committed them with git commit. Finally, I pushed the changes to the GitHub repository.

\begin{verbatim}
git add assignment.tex
git commit -m "Add initial LaTeX file"
git push origin main
\end{verbatim}

\textbf{Setting Up Directory Structure and Additional Files} \\
To better organize the project, I created directories for sections and images and placed some LaTeX files like introduction.tex and git.tex in the sections directory.

\begin{verbatim}
mkdir sections images
touch sections/introduction.tex sections/git.tex
\end{verbatim}

\textbf{Making Regular Commits} \\
After each modification, I tracked and committed the changes using git add and git commit, and then pushed them to GitHub.

\begin{verbatim}
git add sections/git.tex
git commit -m "Added Git section to LaTeX document"
git push origin main
\end{verbatim}

\textbf{Using Tags to Trigger GitHub Actions} \\
To trigger GitHub Actions for automatic LaTeX PDF compilation, I used tags. I created a new tag and pushed it to GitHub.