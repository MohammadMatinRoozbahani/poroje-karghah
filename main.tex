\documentclass{article}

\begin{document}

\begin{center}
    {\Huge be nam khoda} \\[1cm]
    {\LARGE projeye kargah} \\[0.5cm]
    {\large Mohammad Matin Roozbehani} \\[0.5cm]
    {\large 402531387} \\[1cm]
    \rule{\linewidth}{0.4mm}
\end{center}

\vspace{1cm}

\section*{fehrest mataleb}
\begin{itemize}
    \item taklif: inja mitavanid tozihat marboot be taklif ra ezafe konid.
\end{itemize}

\newpage

\section*{taklif}
\textbf{1.1 Repository Initialization and Commits}

\section*{list mataleb}
\textbf{1.1 Repository Initialization and Commits}

\section{Git and GitHub}
\subsection{Repository Initialization and Commits}
\subsection{GitHub Actions for LaTeX Compilation}

\section{Exploration Tasks}
\subsection{Vim Advanced Features}
\subsection{Memory Profiling}
\subsubsection{Memory Leak}
\subsubsection{Memory Profilers}
\subsection{GNU/Linux Bash Scripting}
\subsubsection{fzf}
\subsubsection{Using fzf to Find Your Favorite PDF}
\subsubsection{Opening the File Using Zathura}

\section{Git and FOSS}
\subsection{README.md}
\subsection{Issues}
\subsection{FOSS Contribution}

\end{document}









\section*{taklif}

\textbf{1.1 Repository Initialization and Commits}

\textbf{Setting Up the Repository} \\
To set up the repository, I first went to GitHub and created a new repository named Final-Assignment. During the repository creation, I disabled the Initialize this repository with a README option. After creating the repository, I used the git clone command to clone it to my local system.

\begin{verbatim}
git clone https://github.com/username/Final-Assignment.git
cd Final-Assignment
\end{verbatim}

\textbf{Creating the LaTeX File and Making the First Commit} \\
Inside the local repository, I created a new file called assignment.tex and added initial content. I then added the changes to the staging area using git add and committed them with git commit. Finally, I pushed the changes to the GitHub repository.

\begin{verbatim}
git add assignment.tex
git commit -m "Add initial LaTeX file"
git push origin main
\end{verbatim}
\usepackage{graphicx}
\textbf{Setting Up Directory Structure and Additional Files} \\
To better organize the project, I created directories for sections and images and placed some LaTeX files like introduction.tex and git.tex in the sections directory.

\begin{verbatim}
mkdir sections images
touch sections/introduction.tex sections/git.tex
\end{verbatim}

\textbf{Making Regular Commits} \\
After each modification, I tracked and committed the changes using git add and git commit, and then pushed them to GitHub.

\begin{verbatim}
git add sections/git.tex
git commit -m "Added Git section to LaTeX document"
git push origin main
\end{verbatim}

\textbf{Using Tags to Trigger GitHub Actions} \\
To trigger GitHub Actions for automatic LaTeX PDF compilation, I used tags. I created a new tag and pushed it to GitHub.

\begin{verbatim}
git tag v1.0
git push origin v1.0
\end{verbatim}















\section*{1.2 GitHub Actions for LaTeX Compilation}

\textbf{Steps to Set Up GitHub Actions for Automatic LaTeX Compilation}

To set up GitHub Actions for automatic LaTeX compilation, I followed these steps:

1. First, I created a folder named `.github/workflows` in my GitHub repository.
2. Then, I created a YAML file named `latex.yml` inside this folder to define the GitHub Actions configuration.
3. Inside the `latex.yml` file, I added the following configuration to automatically compile the LaTeX file:

\begin{verbatim}
name: Build LaTeX PDF

on:
  push:
    branches:
      - main
  tags:
    - 'v*'

jobs:
  build:
    runs-on: ubuntu-latest

    steps:
      - name: Checkout repository
        uses: actions/checkout@v2

      - name: Set up TeX Live
        uses: yml2latex/texlive-action@v1
        with:
          texlive-version: "2020"

      - name: Compile LaTeX file
        run: |
          pdflatex -interaction=nonstopmode assignment.tex
          pdflatex -interaction=nonstopmode assignment.tex
      - name: Upload PDF as artifact
        uses: actions/upload-artifact@v2
        with:
          name: assignment-pdf
          path: assignment.pdf
\end{verbatim}

4. After setting up the YAML file, every time I add a new tag to the repository, GitHub Actions automatically runs the LaTeX compilation process and provides the resulting PDF in the "Artifacts" section.

\textbf{Challenges}

- The first challenge I encountered was ensuring the proper installation of the correct version of TeX Live, which required some additional setup.
- Sometimes, errors occurred during the LaTeX file compilation due to missing packages or incorrect settings, requiring further investigation.
- Additionally, after each modification to the LaTeX file, I needed to ensure that the PDF was regenerated properly and that the process ran smoothly without issues.









\section*{Advanced Vim Features}

Here are three advanced and lesser-known Vim features that may not have been covered in class:

\subsection*{1. Vim Sessions (Saving and Restoring State)}
One interesting feature of Vim is the ability to use Sessions, which allows you to save the current state and open files, then restore them at a later time. This is particularly useful when working on large projects or multiple files and you don't want to manually reopen them each time.

To save a session, you can use the following command:
\begin{verbatim}
:mksession! ~/my_session.vim
\end{verbatim}

To restore the session and bring back the previous state:
\begin{verbatim}
vim -S ~/my_session.vim
\end{verbatim}

This method ensures that you can return to exactly where you left off, with the files and settings as they were.

\subsection*{2. Vim Marks (Bookmarking Points)}
Marking different points in your files can be very helpful for quick editing or navigation in long files. With Marks, you can save locations you need and return to them later.

To mark a location, simply use the \texttt{m} command followed by a letter:
\begin{verbatim}
ma
\end{verbatim}
This saves a mark named \texttt{a} at the current position.

To go to a marked point:
\begin{verbatim}
'a
\end{verbatim}