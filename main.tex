\documentclass{article}

\begin{document}

\begin{center}
    {\Huge be nam khoda} \\[1cm]
    {\LARGE projeye kargah} \\[0.5cm]
    {\large Mohammad Matin Roozbehani} \\[0.5cm]
    {\large 402531387} \\[1cm]
    \rule{\linewidth}{0.4mm}
\end{center}

\vspace{1cm}

\section*{fehrest mataleb}
\begin{itemize}
    \item taklif: inja mitavanid tozihat marboot be taklif ra ezafe konid.
\end{itemize}

\newpage

\section*{taklif}
\textbf{1.1 Repository Initialization and Commits}

\section*{list mataleb}
\textbf{1.1 Repository Initialization and Commits}

\section{Git and GitHub}
\subsection{Repository Initialization and Commits}
\subsection{GitHub Actions for LaTeX Compilation}

\section{Exploration Tasks}
\subsection{Vim Advanced Features}
\subsection{Memory Profiling}
\subsubsection{Memory Leak}
\subsubsection{Memory Profilers}
\subsection{GNU/Linux Bash Scripting}
\subsubsection{fzf}
\subsubsection{Using fzf to Find Your Favorite PDF}
\subsubsection{Opening the File Using Zathura}

\section{Git and FOSS}
\subsection{README.md}
\subsection{Issues}
\subsection{FOSS Contribution}

\end{document}









\section*{taklif}

\textbf{1.1 Repository Initialization and Commits}

\textbf{Setting Up the Repository} \\
To set up the repository, I first went to GitHub and created a new repository named Final-Assignment. During the repository creation, I disabled the Initialize this repository with a README option. After creating the repository, I used the git clone command to clone it to my local system.

\begin{verbatim}
git clone https://github.com/username/Final-Assignment.git
cd Final-Assignment
\end{verbatim}

\textbf{Creating the LaTeX File and Making the First Commit} \\
Inside the local repository, I created a new file called assignment.tex and added initial content. I then added the changes to the staging area using git add and committed them with git commit. Finally, I pushed the changes to the GitHub repository.

\begin{verbatim}
git add assignment.tex
git commit -m "Add initial LaTeX file"
git push origin main
\end{verbatim}
